\chapter{On-Chip Network prototypes}
\label{ch:on_chip_networks}

In this chapter, we demonstrate how all of the discussed strategies function to produce computation on hardware in practice.

Closed loop network.
Prototype and demonstrations

\section{Clustered Winner-Take-All}

\subsection{Hardware-friendly aspects of the architecture}

\subsection{Results}

\subsection{Achieveing EI balance in practice. A step-by-step guide.}

With limited control over individual groups of connections in mind, we present a procedure for tuning the chip biases that we used to achieve the expected reproducible behaviour of the network.

\begin{enumerate}
    \item Enable input
    \item Enable EI connections
    \item Enable II connections if needed
    \item Enable IE connections
    \item Enable EE connections
    \item Remove input to check self-sustainability of the bump. Reduce EE connections to the extent of the bump disappearing.
\end{enumerate}

\subsection{Signal processing capabilities}
\subsection{Network capacity}
\subsection{Selective amplification}

\section{Relational networks}

\section{WTA and Plasticity. Unsupervised map learning}

\section{Adaptive target following with reinforcement learning}

Closing the sensorimotor loop with the fully neuromorphic pipeline.

\section{Spatiotemporal pattern recognition prototype with delay lines}

The final network prototype taps into the temporal coding domain while still building on top of the previous rate-coding-based structures. 


\dz{https://ieeexplore.ieee.org/document/5173584}

\section{Discussion: applications and future development}